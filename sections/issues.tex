% vim:textwidth=80:
\section{Issues and Future Work}

The I/O router placement attempts to address the issue of traffic routing and
imbalance at the system level. The route selection algorithm aims to minimize
hops and mitigate contention between the compute clients and I/O routers.
However, several issues remain.

Titan's scheduler is completely oblivious to the placement of I/O routers; jobs
are placed based on node availability.  No mechanism exists for nodes to
request locality to or distance from I/O routers.  A job placed entirely within
one section (two rows) of the machine, for example, will never be able to
achieve greater than $\frac{1}{4}$ of the total file system bandwidth.
Identical jobs placed in different sections of the machine may have widely
varying locality to routers.  To users, this manifests as I/O variability.

Benchmark tests against Spider~II have been run using both optimally placed and
randomly placed clients.  On a quiet system the difference between the two
modes is minimal.  However, on a busy system the difference can be more
substantial.  Arbitrary users jobs have no insight into which ranks are closest
to I/O routers.  To overcome this limitation, OLCF is designing an end-to-end
balanced data placement strategy to complement the backend fine grained routing
algorithm. The primary goal is to extend the balancing from the clients,
through the I/O routers, and into the final destination. This work is ongoing.

While the concepts and algorithms behind fine-grained routing are
straightforward, the actual implementation is quite complex.  When issues
arise, it can be difficult narrow down to find the root cause.  Over time,
various scripts have been developed to ensure all nodes are cabled correctly
and that they can communicate properly.  Additional work improving these
scripts will aid in timely debugging.
