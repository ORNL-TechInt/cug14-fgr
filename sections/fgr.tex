\section{Fine-Grained Routing}

The first step towards achieving higher performance is to place LNET routers
equidistant from each other as much as possible, subject to physical
constraints and other boundary conditions. This ensures that LNET routers are
distributed across the machine and are not segregated in a particular zone. The
second key step is to pair clients with their closest possible LNET routers.
This section discusses the algorithm and implementation that pairs a client
with the optimal LNET routers to minimize hop counts and congestion.

\subsection{Router Selection Algorithm}
Recall from previous sections that there are a total of 440 LNET routers where
only 432 routers are used for file I/O; the remaining 8 are used for metadata
operations. Four LNET routers together form a router module, resulting in a
total of 108 router modules spread throughout the system. These 108 router
modules are divided into 9 groups of 12 modules each, corresponding to the 9
InfiniBand switches present in each row of Spider~II.  Each group is further
divided into 4 sub-groups that service two rows of Titan. Each sub-group
consists of 3 router modules.

Algorithm~\ref{alg:fgr} describes how a client chooses the optimal router
module for a given group. The client-to-group pairing is decided using a fixed
arrangement.  Note that in the presented algorithm $R^G_{i}(S)$ denotes
$i^{th}$ router module in the $G^{th}$ group and $S^{th}$ sub-group. Based on
the number of groups, sub-groups, and router modules in the system, $G$, $S$
and $i$ will have the following respective ranges: ($1, \dots, 9$), ($1, \dots,
4$), and ($1, \dots, 3$).

%Before we describe the algorithm in detail, we firstdescribe how 
%A router module is denoted in our algorithm depending on its association with a particular group and sub-group. The 
%$i^{th}$ router module in the $G^{th}$ group and $S^{th}$ sub-group is denoted as $R^G_{i}(S)$. 

The algorithm takes two input parameters: coordinates of the client ($C$) and
the destination router group ($R^G$). The algorithm returns the coordinates of
three routers assigned to the input client ($C$): one primary and two backup
routers. 

The fine-grained routing algorithm consists of two steps. The first step
(line~\ref{first:begin}--\ref{first:end}) is to choose the sub-group whose
Y-coordinates fall within the close range of Y-coordinates of the input client,
$C$. This is because the bandwidth in the Y-direction is limited compared to
other directions, as discussed earlier. Therefore, it is desirable to minimize
the traffic in that direction first. 

Once the sub-group is selected, the second step
(line~\ref{second:begin}--\ref{second:end}) is to return assigned routers from
this sub-group. As mentioned earlier, each sub-group consists of 3 routers and
the algorithm returns a vector of 3 routers. So, all three of them are
returned, but the one with the lowest distance in X-dimension is assigned as
the primary router to minimize the hops and avoid congestion in that direction.
Note that the X-direction crosses cabinet boundaries, therefore it more
desirable to minimize the hops in that direction compared to the Z-direction
that run within a cabinet in vertical direction.


\begin{algorithm}
\caption{Fine-grained routing algorithm}
\label{alg:fgr}
\begin{algorithmic}[1]
\Procedure {Route Selection Algorithm }{$R^G$, $C$} \\ 
%\hrulefill
\State Divide $R^G$ into 4 sub-groups: $R^G(1)$ \ldots $R^G(4)$.

\ForAll{sub-groups $R^G(i)$} \Comment{$i$ ranges 1 to 4}
\label{first:begin}
    \State $C[y]$ $\leftarrow$ y coordinate of $C$
    \State $R^G_{1}(i)$ $\leftarrow$ first router module in the $i^{th}$ sub-group
    \State $R^G_{1}(i)[y]$ $\leftarrow$ y coordinate of $R^G_{i}(S)$
    \If{ ($C[y] == R^G_{1}(i)[y]-1$) 
    \\ \hspace{\algorithmicindent} \hspace{\algorithmicindent}   or ($C[y] == R^G_{1}(i)[y]$)
    \\ \hspace{\algorithmicindent} \hspace{\algorithmicindent}   or ($C[y] == R^G_{1}(i)[y] + 1$) 
    \\ \hspace{\algorithmicindent} \hspace{\algorithmicindent}   or ($C[y] == R^G_{1}(i)[y] + 2$)}
    \State break with sub-group $i$ selected
    \EndIf
\EndFor
\label{first:end}

\\ 

\State $i$ $\leftarrow$ index of selected sub-group
\label{second:begin}
\State $r_1, r_2, r_3$ $\leftarrow$ first, second, and third router module 
\State \hspace{\algorithmicindent} selected sub-group $i$

\State $d_{min}$ $\leftarrow$ $\infty$ 
\State $Index_{primary}$ $\leftarrow$ $\infty$

\\

\For{$j$ in $1, \dots, 3$}
    \State $d_{current}$ $\leftarrow$ dist($C[x], r_j[x]$) \Comment{distance along $X$ dimension}
        \If{$d_{current}$ $<$ $d_{min}$}
        \State $Index_{primary}$ $\leftarrow$ $j$
        \EndIf

%    \State primary router module $\leftarrow$ min($d_min$)
%    \State backup router modules $\leftarrow$ 
%    \IndState{0} two other modules in the sub-group 
\EndFor

    \State primary router module $\leftarrow$ $R^G_{Index_{primary}}(i)$
    \State backup router modules $\leftarrow$ 
    \IndState{0} two other modules in the $i^{th}$ sub-group

    \Return $<$primary and backup router modules$>$ \\ 

\label{second:end}

\EndProcedure
\\
%\hrulefill
\end{algorithmic}
\end{algorithm}

\subsection{LNET Routing}

A quick introduction to LNET routing is helpful to understanding Titan's setup.
Lustre uses LNET for all communications between clients, routers, and servers.
Routing allows communication to span multiple network types, as long as one or
more nodes exist that can ``bridge'' the disjoint networks.  Each unique
network is given an identifying name that consists of the network type and an
arbitrary integer (for example, \textit{o2iblnd0} for the $0^{th}$ InfiniBand
network or \textit{gni101} for the $101^{st}$ Gemini network).

Each node in an LNET network has a unique Network Identifier (NID) that is in
the form \textit{identifier@network}.  The InfiniBand Lustre Networking Driver
(LND) uses the IP-over-IB address as its unique identifier
(ex.~\textit{10.10.10.101@o2iblnd0}), while the Gemini LND uses its Gemini
network ID (ex.~\textit{4044@gni101}).  It is permissable for a network
interface to have multiple NIDs assigned to it.  For example, a node with a
single InfiniBand interface may have NIDs \textit{10.10.10.101@o2ib0},
\textit{10.10.10.101@o2ib1}, and \textit{10.10.10.101@o2ib2}.  The Lustre
Manual \cite{lustre-manual} describes how to specify network settings.

In mixed-network environments, system administrators setup the LNET routing
tables on each node.  For every remote network that the node should be able to
communicate with, a list of routers should be provided.  Each router is given a
``hop count'' that indicates the relative distance between the nodes.

When a packet is ready to leave a node, the destination network ID is compared
to the list of local network IDs.  If a match is found, then the message is
sent directly to the destination.  Otherwise, the routing table must be used to
determine the next hop.  Under normal circumstances, LNET will cycle through
all the appropriate routers with the lowest hop count.  Routers with higher hop
counts will only be used if \textbf{all} routers of a lower hop count are
unavailable.

\subsection{FGR in Practice}

Thirty-seven IB LNET network interfaces (NIs) exist that correspond to the 36
IB leaf switches, plus one for metadata and DVS traffic. Each LNET router has
exactly one IB NI that corresponds to the switch to which it connects. The
service and compute nodes are broken up into twelve Gemini regions based on
their topological location. The clients use this Gemini NI as the source
address so the Lustre servers return the I/O to a topologically close router in
the system. Several scripts were developed to calculate the appropriate LNET
NIs (both IB and Gemini), setup the routes, and verify that the routers are
cabled correctly.
%Zones in the network

%Return traffic routing

%FGR scripts

% vim:textwidth=80:
