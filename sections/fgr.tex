\section{Fine-Grained Routing}

With router placement in place, we now face the question that, how does each
client select which router to minimize communication hops as well as to avoid
congestion. This section provides an overview on route selection process and
algorithms. Routers are divided into 9 groups, with each group containing 12
router modules. We denote a router group with a superscript, and a router
module in that group with a subscript. For example, the first module in a
router group $A$ is denoted as $R^A_1$ so on and so forth. The following
algorithm describe the core idea of selection process. The input is a given
router group and client ID, the output is a triple: primary router and two back
up routers.

\begin{algorithmic}[1]
\Procedure {Route Selection Algorithm }{$R^x$, $C$} \\ \hrulefill

\State Divide $R^x$ into 4 sub-groups: $R^x(1)$ \ldots $R^x(4)$.

\ForAll{sub-groups $R^x(i)$}
    \State $C_y$ $\leftarrow$ y coordinate of $C$
    \State $R$ $\leftarrow$ first router module in current sub group
    \State $R_y$ $\leftarrow$ y coordinate of $R$
    \If{$C_y == R_y-1$ or $C_y == R_y$ or \\
       \hspace{\algorithmicindent} \hspace{\algorithmicindent} $C_y == R_y + 1$ or $C_y == R_y + 2$}
    \State break with subgroup $i$ selected
    \EndIf
\EndFor

\State $i$ $\leftarrow$ index of selected sub group
\State $r_1, r_2, r_3$ $\leftarrow$ first, second, and third router module 
\State \hspace{\algorithmicindent} selected subgroup $i$

\ForAll{$r_1, r_2, r_3$}
    \State $d_i$ $\leftarrow$ dist($C, r_i$) \Comment{distance along $X$ dimension}
    \State primary router module $\leftarrow$ min($d_i$)
\EndFor

\EndProcedure
\\\hrulefill
\end{algorithmic}


%Zones in the network

%Return traffic routing

%FGR scripts

% vim:textwidth=80:
